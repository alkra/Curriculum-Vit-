\newcommand\IGN{%
  \foreignlanguage{french}{Institut géographique national}%
}

\newcommand\tuwiendate{mai -- novembre 2016}
\newcommand\tuwiencont{%
  \work{Stage} (6 mois):
  «~Réalisation d'une application web côté client sur smartphone~»

  {\small Technische Universität Wien, Geoinformation
    Forschungsgruppe}
}
\newcommand\tuwienlong{
  \begin{itemize}
  \item amélioration de mes connaissances en développement web ;
  \item stockage des données côté client ;
  \item documentation et publication des problèmes rencontrés et des
    solutions apportées.
  \end{itemize}
}

\newcommand\structdate{hiver \linebreak[1] 2015 -- 2016}

\newcommand\openglTdate{automne 2015}
\newcommand\openglTcont{%
  \href{https://github.com/clemousse/megafi}{%
    \work{Projet de développement}%
  } (20 jours):
  «~Modèle d'écoulement sur un MNT~»

  \begin{itemize}
  \item développement en Qt (C++) ;
  \item visualisation en OpenGL ;
  \end{itemize}
}

\newcommand\geoconceptdate{été 2015}
\newcommand\geoconceptcont{%
  \work{Stage} (11 semaines):
  «~Un algorithme pour calculer la médiane d'un graphe~»

  {\small Groupe GEOCONCEPT, département R\,\&\,D, Bagneux (92)}
}
\newcommand\geoconceptlong{%
  \begin{itemize}
    \itemsep1pt \parsep0pt \parskip0pt
  \item bibliographie ;
  \item conception et implémentation en C++ d'un algorithme ;
  \item mesure de performances en temps et consommation mémoire ;
  \item conception d'une interface graphique ;
  \item méthode agile SCRUM.
  \end{itemize}
}

\newcommand\developementdate{printemps 2015}
\newcommand\developementcont{%
  \href{https://github.com/alkra/BenchOKR}{\work{Projet
      développement}} (14 jours)
  sur le sujet du rendu de nuages de points 3D

  {\small Laboratoire VALILAB,
    \IGN,
    Marne-la-Vallée (77)}
}

\newcommand\researchdate{hiver \linebreak[1] 2014 -- 2015}
\newcommand\researchcont{%
  \work{Projet recherche} (15 jours)
  en cartographie

  {\small Laboratoire COGIT,
    \IGN,
    Saint-Mandé (94)}
}

\newcommand\forcalquierdate{été 2014}
\newcommand\forcalquiercont{%
  \work{Stage terrain} (10 semaines)

  {\small\IGN,
    Forcalquier (04)}
}
\newcommand\forcalquierlong{%
  \begin{itemize}
    \itemsep1pt \parsep0pt \parskip0pt
  \item mesures au tachéomètre et au GPS ;
  \item fabrication d'un modèle 3D d'une chapelle ;
  \item petit projet de recherche en géologie.
  \end{itemize}
}

%%% Local Variables:
%%% mode: latex
%%% TeX-master: "../cv"
%%% End: