Je suis étudiant à l'École nationale des sciences géographiques (ENSG) de Marne-la-Vallée, en deuxième année du cycle d'ingénieur. À la fin de cette année, je dois réaliser un stage d'au moins onze semaines à partir de fin mai jusqu'à mi-août. Mon école m'a fait parvenir cette offre de stage, et je m'y intéresse tout particulièrement.

Ma formation en géomatique est polyvalente : j'étudie les techniques d'acquisition, la cartographie et l'analyse spatiale, les problématiques modernes de la ville et du développement durable, avec aussi beaucoup d'informatique. En classes préparatoires, j'ai beaucoup apprécié les enseignements d'informatique, très théoriques. Ces connaissances ont enrichi les cours qui m'ont été dispensés à l'école, davantage orientés vers la pratique à l'exception du cours d'algorithmique et de théorie des graphes. J'envisage de me spécialiser en informatique générale l'année prochaine, et un stage dans ce domaine serait particulièrement motivant, en plus d'être parfaitement approprié.

Deux projets en équipe réalisés à l'école m'ont préparé au contexte de ce stage. Le premier a été réalisé dans un laboratoire de recherche en cartographie, et consistait en une étude bibliographique puis pratique de la représentation des itinéraires de randonnée ; il fut une excellente approche du monde de la recherche, et fournit une expérience connexe au stage. Le deuxième est un projet de développement informatique, en C\raisebox{1.5pt}{\small $++$} et OpenGL, qui a été encadré en méthode agile ; j'en ai été le chef de projet, ce qui m'a placé idéalement pour observer le comportement d'une petite équipe.

Il semble que le travail présenté dans cette offre de stage correspond parfaitement à la formation délivrée par mon école, et de plus dans un domaine qui m'intéresse particulièrement. C'est donc avec plaisir que je prendrai le temps de venir vous rencontrer, en espérant ne pas avoir à manquer un cours.

Dans l'attente de votre réponse, je vous prie, Madame, Monsieur, d'agréer mes salutations sincères.


%%% Local Variables:
%%% mode: latex
%%% TeX-master: "../cv.tex"
%%% End:
