%%  Alban KRAUS
%%  December 6th, 2015
%%
%%  Curriculum Vitæ
%%  Version: General
%%
%%  The content of this work (the Content) is made of all .tex files
%%  of this repository, as well as the images located in the imgs
%%  subfolder. It is protected under the French law. Permission is
%%  granted to:
%%
%%  * download and read the Content; you may make minor modifications
%%    in order to make the files readable;
%%
%%  * generate a compiled document from the unmodified Content;
%%
%%  * modify all or part of the Content, thus creating a derivative
%%    work; in that case, you must remove all content that refer to
%%    me, such as my name, photograph, and other personal data. This
%%    permission does not apply to the photographs in imgs folder;
%%    thus, you must remove them from your derivative work.
%%
%%  You may distribute unmodified copies of the Content. You may
%%  distribute a compiled document if it respects the semantics and
%%  disposition of the output of pdflatex running on the unmodified
%%  Content. In any case, you are not allowed to make commercial
%%  benefits from the distribution. If you distribute any of the
%%  preceding, you must not apply terms that restrict or extend the
%%  permissions you have been granted.
%%
%%  Alternatively, you can reuse the Content under the terms of the
%%  Creative Commons – Attribution – Non-commercial – No derivatives
%%  licence.

\documentclass[a4paper, oneside, 11pt, draft]{article}

\usepackage[utf8]{inputenc} % Setting the encoding of the file

\newcommand\imgdir{imgs/}
\newcommand\lang{en}

\newif\ifcv
\newif\ifletter

\cvtrue
\letterfalse

\usepackage[\lang]{cv}

\newlength{\englishbox} % For the TOEIC parbox
\settowidth{\englishbox}{\textbf{English}}
\newlength{\frenchbox}
\settowidth{\frenchbox}{Mother tongue}
\newlength{\germanbox}
\settowidth{\germanbox}{Proficient}


% Adding color to my document
\definecolor{edu}{rgb}{0, 0, 0}
\newcommand{\education}[1]{\textbf{\color{edu}\large #1}}
\definecolor{work}{rgb}{0, 0, 0}
\newcommand{\work}[1]{\textbf{\color{work}\large #1}}
\definecolor{english}{rgb}{0, 0, 0.2}
\newcommand{\englishcolor}{\color{english}}

%%%%%%
%%%%%%
\begin{document}

%%%%%%
% CV %
%%%%%%
\ifcv

\newlength\imgwidth
\newlength\imgheight
\newlength\nameWidth
\newlength\addressWidth
\newlength\photoWidth
\newcommand\photoRatio{}
\setlength\imgheight{10em}
\setlength\imgwidth{0.820408163\imgheight}

\ifphoto
\setlength\nameWidth{0.45\textwidth}
\setlength\addressWidth{0.3\textwidth}
\renewcommand\photoRatio{0.25}
\else
\setlength\nameWidth{0.5\textwidth}
\setlength\addressWidth{0.5\textwidth}
\renewcommand\photoRatio{0}
\fi
\setlength\photoWidth{\photoRatio\textwidth}

\newcommand\imgname{%
  \ifprinted%
    bon-profil%
  \else%
    bon-profil-leger%
  \fi%
}

\setlength\tabcolsep{0pt}

\begin{tabular}{@{}p{\nameWidth}@{}p{\addressWidth}@{}p{\photoWidth}@{}}
  \newcommand\piMail{%
  \raisebox{-1pt}{%
    \includegraphics[width=1em, height=1em, keepaspectratio=true]{\imgdir Mail}%
  }%
}%
%
\newcommand\piMobile{%
  \raisebox{-1.5pt}{%
    \includegraphics[width=1em, height=1em, keepaspectratio=true]{\imgdir Mobile}%
  }%
}%
%
\newcommand\piEmail{%
  \textbf{@}%
}%
%
\begin{tabularx}{\textwidth}{@{}d X@{}}
  \textbf{Mr} & {\Large\textbf{Alban KRAUS}}\\

  born & 4/6/1993\\[2pt]

  \piMail & \foreignlanguage{french}{Le Bois Grand}\\[-1.5pt]
  & \foreignlanguage{french}{19330 Chameyrat}\\[-1.5pt]

  \piMobile & (33) 6 35 94 54 03\\

  \vspace{3pt}

  \piEmail & \Href{mailto:alban.kraus@gmail.com}{\mbox{alban.kraus@gmail.com}}\\
\end{tabularx}

%%% Local Variables:
%%% mode: latex
%%% TeX-master: "../cv"
%%% End:
  &
    \ifphoto
    \setlength\unitlength{\textwidth}%
    \begin{picture}(\photoRatio,0)
      \newsavebox{\image}
      \savebox{\image}(\photoRatio, 0)[tr]{%
        \includegraphics[width=\imgwidth, height=\imgheight, keepaspectratio=true]{\imgdir\imgname}%
      }%
      \put(0, 0.05){\usebox{\image}}
    \end{picture}%
    \fi
\end{tabular}

\setlength\tabcolsep{\defaulttabcolsep}

\vspace{-1em}
\begin{center}\Large
  Ingénieur en géomatique
\end{center}

%%% Local Variables:
%%% mode: latex
%%% TeX-master: "cv"
%%% End:


\cvtitle{Main skills}

\makebox[\textwidth][s]{
  \mbox{Patient}
  \mbox{Methodical}
  \mbox{Geographic information}
  \mbox{Software development}
%  \mbox{Web \& database}
}

%%% Local Variables:
%%% mode: latex
%%% TeX-master: "../cv"
%%% End:

\newlength\arrayskip
\setlength\arrayskip{3pt}
\renewcommand{\arraystretch}{1.5}
\newcommand\content[1]{%
  \parbox{\tablecontw}{%
    \setlength\parskip{2pt}%
    \vspace{\arraystretch\arrayskip}%
    #1%
    \vspace{\arraystretch\arrayskip}%
  }%
}
\setlength\parskip{2pt}

\cvtitle{Éducation}

\begin{tabu} to \linewidth{@{}d X@{}}
  2013 -- 2016 & \content{%
    \education{Diplôme d'ingénieur en géomatique},
    spécialisé en systèmes d'information

    \href{http://www.ensg.eu}{%
      École nationale des sciences géographiques (ENSG Géomatique)%
    }, Marne-la-Vallée (77)%
  }\\

  2011 -- 2013 & \content{%
    Classe préparatoire aux grandes écoles

    Lycée du Parc, Lyon%
  }\\

  2011 & \education{Baccalauréat} scientifique,
  mention \emph{Très Bien} (17 / 20)\\
\end{tabu}

%%% Local Variables:
%%% mode: latex
%%% TeX-master: "../cv"
%%% End:

\newcommand\IGN{%
  \foreignlanguage{french}{Institut géographique national}%
}

\newcommand\structdate{winter \linebreak[1] 2015 -- 2016}

\newcommand\openglTdate{fall 2015}

\newcommand\geoconceptdate{summer 2015}
\newcommand\geoconceptcont{%
  11-week
  \work{internship}:
  ``An algorithm to compute the median of a graph''

  {\small GEOCONCEPT group, R\,\&\,D department, Paris}

  \begin{itemize}
    \itemsep1pt \parsep0pt \parskip0pt
  \item gathered a bibliography
  \item designed and implemented an algorithm
  \item measured time and memory consumption
  \item designed a graphical user interface
  \end{itemize}
}

\newcommand\developementdate{spring 2015}
\newcommand\developementcont{%
  14-day
  \work{Development project}
  on 3D point clouds rendering

  {\small VALILAB laboratory,
    \IGN,
    Paris}
}

\newcommand\researchdate{winter \linebreak[1] 2014 -- 2015}
\newcommand\researchcont{%
  15-day
  \work{Research project}
  in Cartography

  {\small COGIT laboratory,
    \IGN,
    Paris}
}

\newcommand\forcalquierdate{summer 2014}
\newcommand\forcalquiercont{%
  10-week internship in
  \work{Surveying}

  {\small\IGN,
    Forcalquier (South of France)}

  \begin{itemize}
    \itemsep1pt \parsep0pt \parskip0pt
  \item performed measurements with a tacheometer and a GPS receiver
  \item created a 3D model of a chapel
  \item carried out a short research in geology
  \end{itemize}
}

%%% Local Variables:
%%% mode: latex
%%% TeX-master: "cv"
%%% End:

\newif\iftuwien       % IT3 internship
\newif\iftuwienlong
\newif\ifstruct       % IT3 structuration project
\newif\ifopenglT      % opengl IT3 project
\newif\ifgeoconcept   % IT2 internship
\newif\ifgeoconceptlong
\newif\ifdevelopement % IT2 developement project
\newif\ifresearch     % IT2 research project
\newif\ifforcalquier  % IT1 Forcalquier internship
\newif\ifforcalquierlong

\tuwientrue
\tuwienlongtrue
\structfalse
\openglTfalse
\geoconcepttrue
\geoconceptlongtrue
\developementfalse
\researchtrue
\forcalquiertrue
\forcalquierlongfalse

\cvtitle{Projects and Internships}

\begin{tabu} to \linewidth {@{}d X@{}}
  \iftuwien
  \tuwiendate & \content{%
    \tuwiencont
    \iftuwienlong
      \tuwienlong%
    \fi%
  }\\
  \fi

  \ifstruct
  \fi

  \ifopenglT
  \openglTdate & \content{\openglTcont}\\
  \fi

  \ifgeoconcept
  \geoconceptdate & \content{%
    \geoconceptcont
    \ifgeoconceptlong
      \geoconceptlong%
    \fi%
  }\\
  \fi

  \ifdevelopement
  \developementdate & \content{\developementcont}\\
  \fi

  \ifresearch
  \researchdate & \content{\researchcont}\\
  \fi

  \ifforcalquier
  \forcalquierdate & \content{%
    \forcalquiercont
    \ifforcalquierlong
      \forcalquierlong%
    \fi%
  }\\
  \fi
\end{tabu}

%%% Local Variables:
%%% mode: latex
%%% TeX-master: "../cv"
%%% End:

\renewcommand\arraystretch{1.2}
%\setlength\parskip{\defaultparskip}

\cvtitle{Computer skills}

\begin{tabu} to \linewidth {@{}d X@{}}
  Languages & \textbf{C$++$}, Web, XML, Python, Java, UML\\
  Libraries & \textbf{Qt}, \textbf{OpenGL}, J2EE, Android, OpenMP\\
  GIS & \textbf{PostgreSQL} + PostGIS, QGIS, ArcGIS\\
  Systems & \textbf{GNU/Linux}, Docker
\end{tabu}

%%% Local Variables:
%%% mode: latex
%%% TeX-master: "../cv"
%%% End:

\renewcommand\arraystretch{1}
%\setlength\parskip{\defaultparskip}

\vspace{-1em}
\begin{tabularx}{\textwidth}{@{}p{0.5\textwidth} X@{}}
  \cvtitle{Languages}

\begin{tabularx}{\textwidth}{@{}d X@{}}
  \textbf{French} & Mother tongue\\
  \textbf{English} & Fluent (\textcolor{english}{TOEIC 2014: $950 / 990$})\\
  \textbf{German} & Proficient\\
\end{tabularx}

%\vspace{0.5pt}
%
%{\small
%  Latin: 7 years\\
%  Ancient Greek: 4 years}

%%% Local Variables:
%%% mode: latex
%%% TeX-master: "../cv"
%%% End:
  &
  \cvtitle{Autres activités}

\begin{itemize}[leftmargin=1em]
  \itemsep0pt \parsep0pt \parskip0pt
% \item karaté, 2003--2008 ;
\item \textbf{organiste} \textsl{(depuis l'âge de 9 ans)}
  % :
  % \begin{itemize}
  %   \itemsep0pt \parsep0pt \parskip0pt
  % \item accompanied several masses
  % \item took part in concerts (J.\/S.~Bach)
  % \item good level in music reading
  % \end{itemize}
  ;
\item \textbf{rédacteur en chef} du journal des étudiants, 2014.
\end{itemize}

%%% Local Variables:
%%% mode: latex
%%% TeX-master: "../cv"
%%% End:

  % \vspace{1.8em}

  % Noisy-le-Grand, 21 décembre 2015

\ifprinted
\else
  \centering
  \includegraphics[width=0.4\textwidth]{\imgdir Sign}
\fi

%%% Local Variables:
%%% mode: latex
%%% TeX-master: "../cv"
%%% End:
\end{tabularx}

%%% Local Variables:
%%% mode: latex
%%% TeX-master: "cv"
%%% End:

\ifletter
  \clearpage
\fi
\fi


%%%%%%%%%%
% Letter %
%%%%%%%%%%
\ifletter

\newgeometry{top=3cm, bottom=3cm, left=3.5cm, right=3.5cm}
\setlength{\parskip}{0.3\defaultparskip}

\setlength{\tabcolsep}{0pt}
\begin{tabularx}{\textwidth}{l r}
\parbox{0.5\textwidth}{
  \setlength{\parskip}{0.3\defaultparskip}
  Alban Kraus

50, avenue Médéric\\
93\/160 Noisy-le-Grand\\
France

\href{mailto:alban.kraus@ensg.eu}{alban.kraus@ensg.eu}

} & \parbox{0.5\textwidth} {
  \setlength{\parskip}{0.3\defaultparskip}
  \raggedleft
  \input{letter/2-recipient}
}\\
\end{tabularx}

\bigskip

\begin{flushright}
  Noisy-le-Grand, \today
\end{flushright}

\textbf{Subject:}
\input{letter/3-object}

\bigskip
\setlength{\parskip}{\defaultparskip}

Madame, Monsieur,

Je suis étudiant à l'École nationale des sciences géographiques (ENSG) de Marne-la-Vallée, en deuxième année du cycle d'ingénieur. À la fin de cette année, je dois réaliser un stage d'au moins onze semaines à partir de fin mai jusqu'à mi-août. Mon école m'a fait parvenir cette offre de stage, et je m'y intéresse tout particulièrement.

Ma formation en géomatique est polyvalente : j'étudie les techniques d'acquisition, la cartographie et l'analyse spatiale, les problématiques modernes de la ville et du développement durable, avec aussi beaucoup d'informatique. En classes préparatoires, j'ai beaucoup apprécié les enseignements d'informatique, très théoriques. Ces connaissances ont enrichi les cours qui m'ont été dispensés à l'école, davantage orientés vers la pratique à l'exception du cours d'algorithmique et de théorie des graphes. J'envisage de me spécialiser en informatique générale l'année prochaine, et un stage dans ce domaine serait particulièrement motivant, en plus d'être parfaitement approprié.

Deux projets en équipe réalisés à l'école m'ont préparé au contexte de ce stage. Le premier a été réalisé dans un laboratoire de recherche en cartographie, et consistait en une étude bibliographique puis pratique de la représentation des itinéraires de randonnée ; il fut une excellente approche du monde de la recherche, et fournit une expérience connexe au stage. Le deuxième est un projet de développement informatique, en C\raisebox{1.5pt}{\small $++$} et OpenGL, qui a été encadré en méthode agile ; j'en ai été le chef de projet, ce qui m'a placé idéalement pour observer le comportement d'une petite équipe.

Il semble que le travail présenté dans cette offre de stage correspond parfaitement à la formation délivrée par mon école, et de plus dans un domaine qui m'intéresse particulièrement. C'est donc avec plaisir que je prendrai le temps de venir vous rencontrer, en espérant ne pas avoir à manquer un cours.

Dans l'attente de votre réponse, je vous prie, Madame, Monsieur, d'agréer mes salutations sincères.


%%% Local Variables:
%%% mode: latex
%%% TeX-master: "../cv.tex"
%%% End:


\begin{flushleft}
  \ifprinted
  \else
    \includegraphics[width=0.25\textwidth]{\imgdir Sign}\par
  \fi
  Alban Kraus
\end{flushleft}

\fi
\end{document}